\documentclass{article}
\usepackage[utf8]{inputenc}
\usepackage{amsmath}
\usepackage{algorithm}
\usepackage{algpseudocode}
\usepackage{float}
\usepackage[margin=2.2cm]{geometry}
\usepackage{hyperref}

\title{Theory Assignment-3: ADA Winter-2024}
\author{Himang Chandra Garg (2022214) \and Nishil Agarwal (2022334)}

\date{}
\begin{document}

\maketitle

\section{Preprocessing}
% Your preprocessing steps go here
Not Applicable in this Algorithm.


\section{Assumption}
One cannot rotate a block of marble. Let's say the cost of the (i,j) size block is x, and for (j, i) size it is y. Then, a block of size height = i, width = j is sold for x only. It cannot be sold for y by rotating.


\section{Subproblem}
We calculate this value at each step:\\\\
dp[h][i] = The maximum profit that can be obtained by selling (h+1)*(i+1) centimeters of the slab.
\\\\
Here, h+1 represents the height of the current marble slab, and i+1 represents the width of the current marble slab.
\\
We have started counting h and i from 0 to m-1 and n-1, respectively. The actual height and width of a slab of dimensions (h, i) is (h+1,i+1).


\section{Reccurrence of Subproblem }
$dp[h][i] = \max \left\{
\begin{array}{ll}
v[h][i]\\
dp[k][i]+dp[h - k - 1][i]\\
dp[h][j]+dp[h][i - j - 1]\\
\end{array}
\right\}$
\\\\\\
Here, j is varying from 0 to i-1. k is varying from 0 to h-1.\\
Hence, $dp[k][i]+dp[h - k - 1][i]$ represents h different terms. Similarly, $dp[h][j]+dp[h][i - j - 1]$ represents i different terms.\\
We are taking the maximum of all of these terms combined together.


\section{Subproblem that solves the final problem }
$\max \left\{
\begin{array}{ll}
v[m-1][n-1]\\
dp[k][n-1]+dp[m - k - 2][n-1]\\
dp[m-1][j]+dp[m-1][n - j - 2]\\
\end{array}
\right\}$
\\\\\\
Here, j is varying from 0 to n-1. k is varying from 0 to m-1.\\
Hence, $dp[k][i]+dp[h - k - 1][i]$ represents m different terms. Similarly, $dp[h][j]+dp[h][i - j - 1]$ represents n different terms.\\
We are taking the maximum of all of these terms combined together.


\section{Algorithm Description}
% Describe your algorithm here
We have used tabularization or a bottom-up approach of Dynamic Programming using a 2-D array.\\

Here, dp[h][i] stores the maximum price that we can get by selling a slab of size (h+1)*(i+1).\\

We loop through m height and n width to access all cells of the dp table. At each iteration, we populate dp[h][i] with the maximum money gainable from selling marble of size (h+1)*(i+1).\\\\

We calculate the maximum price of (h+1)*(i+1) size by the following cases:
\\\\
1.  Either the spot price for (h+1)*(i+1) size slab as a whole is the best price we can get by selling the slab.\\
2.  Or we can sell it in same way as we would sell a (k+1)*(i+1) and a ((h+1)-(k+1))*(i+1) size slab. We would be selling the same (h+1)*(i+1) slab but in a different way. Here k will vary from 0 to h-1 to cover all cases.\\
3.  Or we can sell it in same way as we would sell a (h+1)*(j+1) and a (h+1)*((i+1)-(j+1)) size slab. We would be selling the same (h+1)*(i+1) slab but in a different way. Here j will vary from 0 to k-1 to cover all cases.\\\\

All of these possible cases are taken into account for the calculation of the max price for [h][i]\textsuperscript{th} cell of dp table. We calculate the max out of these h+i+1 options for this.\\

We will continue populating the dp till [m-1][n-1]\textsuperscript{th} cell representing the max sell price of slab of dimensions (m)*(n). This will be our answer.
\\

\section{Runtime Complexity Analysis}
% Analyze the time and space complexity of your algorithm
The time complexity of this Algorithm is:
\begin{center}
O(mn(m+n))
\end{center}
Here, m is the height of the slab, and n is the width of the slab initially given to us.\\
\\We came to this result since our code contains nested operations:\\
1.  Topmost for loop that traverses h from 0 to m-1(m operations).\\
2.  Second level for loop that traverses i from 0 to n-1(n operations).\\
3.  There are 2 bottom-level loops that generate i+h cases for calculating the maximum selling price. Also, the max function used to find the maximum price out of these i+h+1 cases has a time complexity of O(n+m).\\\\
Since the operations are nested, the time complexity gets multiplied. \\
The resulting time complexity in terms of (m+n) can also be written as:
\begin{center}
O((m+n)\textsuperscript{3})\\
\end{center}
Since Big O gives only the upper bound.
\\\\\\\\\\\\\\\\\\\
\section{Pseudocode}
\begin{algorithm}
\caption{Maximize Profit Algorithm}\label{maximize_profit}
\begin{algorithmic}[1]
\Procedure{MaximizeProfit}{$m, n, v$} \Comment{$v$ is a 2D array representing spot prices}
    \State $dp \gets$ Initialize a 2D array of size $m \times n$ with all elements set to $-1$
    \For{$h \gets 0$ \textbf{to} $m-1$}
        \For{$i \gets 0$ \textbf{to} $n-1$}
            \State $temp \gets$ Empty list
            \State Append $v[h][i]$ to $temp$
            \For{$k \gets 0$ \textbf{to} $h-1$}
                \State Append $dp[k][i] + dp[h-k-1][i]$ to $temp$
            \EndFor
            \For{$j \gets 0$ \textbf{to} $i-1$}
                \State Append $dp[h][j] + dp[h][i-j-1]$ to $temp$
            \EndFor
            \State $dp[h][i] \gets$ Maximum value in $temp$
        \EndFor
    \EndFor
    \State \textbf{Output} $dp[m-1][n-1]$
\EndProcedure
\end{algorithmic}
\end{algorithm}
\end{document}
